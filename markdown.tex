% Options for packages loaded elsewhere
\PassOptionsToPackage{unicode}{hyperref}
\PassOptionsToPackage{hyphens}{url}
%
\documentclass[
]{article}
\usepackage{amsmath,amssymb}
\usepackage{iftex}
\ifPDFTeX
  \usepackage[T1]{fontenc}
  \usepackage[utf8]{inputenc}
  \usepackage{textcomp} % provide euro and other symbols
\else % if luatex or xetex
  \usepackage{unicode-math} % this also loads fontspec
  \defaultfontfeatures{Scale=MatchLowercase}
  \defaultfontfeatures[\rmfamily]{Ligatures=TeX,Scale=1}
\fi
\usepackage{lmodern}
\ifPDFTeX\else
  % xetex/luatex font selection
\fi
% Use upquote if available, for straight quotes in verbatim environments
\IfFileExists{upquote.sty}{\usepackage{upquote}}{}
\IfFileExists{microtype.sty}{% use microtype if available
  \usepackage[]{microtype}
  \UseMicrotypeSet[protrusion]{basicmath} % disable protrusion for tt fonts
}{}
\makeatletter
\@ifundefined{KOMAClassName}{% if non-KOMA class
  \IfFileExists{parskip.sty}{%
    \usepackage{parskip}
  }{% else
    \setlength{\parindent}{0pt}
    \setlength{\parskip}{6pt plus 2pt minus 1pt}}
}{% if KOMA class
  \KOMAoptions{parskip=half}}
\makeatother
\usepackage{xcolor}
\usepackage[margin=1in]{geometry}
\usepackage{graphicx}
\makeatletter
\newsavebox\pandoc@box
\newcommand*\pandocbounded[1]{% scales image to fit in text height/width
  \sbox\pandoc@box{#1}%
  \Gscale@div\@tempa{\textheight}{\dimexpr\ht\pandoc@box+\dp\pandoc@box\relax}%
  \Gscale@div\@tempb{\linewidth}{\wd\pandoc@box}%
  \ifdim\@tempb\p@<\@tempa\p@\let\@tempa\@tempb\fi% select the smaller of both
  \ifdim\@tempa\p@<\p@\scalebox{\@tempa}{\usebox\pandoc@box}%
  \else\usebox{\pandoc@box}%
  \fi%
}
% Set default figure placement to htbp
\def\fps@figure{htbp}
\makeatother
\setlength{\emergencystretch}{3em} % prevent overfull lines
\providecommand{\tightlist}{%
  \setlength{\itemsep}{0pt}\setlength{\parskip}{0pt}}
\setcounter{secnumdepth}{-\maxdimen} % remove section numbering
\usepackage{bookmark}
\IfFileExists{xurl.sty}{\usepackage{xurl}}{} % add URL line breaks if available
\urlstyle{same}
\hypersetup{
  pdftitle={Analyse interactive de l'obésité avec R Shiny},
  pdfauthor={{[}Ton Nom{]}},
  hidelinks,
  pdfcreator={LaTeX via pandoc}}

\title{Analyse interactive de l'obésité avec R Shiny}
\author{{[}Ton Nom{]}}
\date{2025-06-13}

\begin{document}
\maketitle

\section{Introduction}\label{introduction}

Ce projet vise à analyser un jeu de données sur les habitudes de vie et
l'obésité afin d'identifier les comportements associés à un risque
accru. Il s'appuie sur un tableau de données public intitulé
\textbf{ObesityDataSet\_raw\_and\_data\_sinthetic.csv} et utilise
\textbf{R Shiny} pour créer un tableau de bord interactif permettant
l'exploration dynamique des résultats.

\section{1. Exploration et préparation des
données}\label{exploration-et-pruxe9paration-des-donnuxe9es}

Les variables principales incluent : - \texttt{Age}, \texttt{Gender},
\texttt{Height}, \texttt{Weight} → données personnelles - \texttt{BMI}
(calculé) → indicateur principal - \texttt{FAVC}, \texttt{CAEC},
\texttt{CALC} → habitudes alimentaires - \texttt{FAF} → activité
physique - \texttt{MTRANS} → moyen de transport - \texttt{NObeyesdad} →
classe d'obésité (cible)

\subsection{Nettoyage appliqué :}\label{nettoyage-appliquuxe9}

\begin{itemize}
\tightlist
\item
  Suppression des valeurs aberrantes sur \texttt{Age}, \texttt{Height},
  \texttt{Weight}, \texttt{BMI}
\item
  Transformation des variables qualitatives en facteurs
\item
  Ajout d'une colonne \texttt{AgeGroup} pour regrouper les âges
\end{itemize}

\section{2. Objectif du dashboard
Shiny}\label{objectif-du-dashboard-shiny}

Le dashboard permet de : - Filtrer les données par \texttt{Genre} et
\texttt{Tranche\ d\textquotesingle{}âge} - Visualiser les différences
entre classes d'obésité selon : - Les habitudes de grignotage
(\texttt{CAEC}) - L'activité physique (\texttt{FAF}) - Le moyen de
transport (\texttt{MTRANS})

\section{3. Visualisations incluses}\label{visualisations-incluses}

\begin{itemize}
\tightlist
\item
  Histogramme de la répartition des classes d'obésité
  (\texttt{NObeyesdad})
\item
  Graphique en barre sur les habitudes alimentaires (\texttt{CAEC} vs
  obésité)
\item
  Boxplot de l'activité physique (\texttt{FAF} vs obésité)
\item
  Graphique en barre des moyens de transport par classe
\end{itemize}

\section{4. Insights observés}\label{insights-observuxe9s}

\begin{itemize}
\tightlist
\item
  Une fréquence de grignotage plus élevée est associée à des niveaux
  plus sévères d'obésité
\item
  L'activité physique est généralement plus basse chez les individus
  obèses
\item
  Les personnes utilisant le transport public semblent davantage
  concernées par le surpoids
\end{itemize}

\section{5. Conclusion}\label{conclusion}

Ce dashboard interactif met en lumière plusieurs corrélations
comportementales liées à l'obésité. Il constitue un outil utile pour les
analystes ou professionnels de santé souhaitant explorer rapidement les
profils à risque.

\section{6. Captures d'écran (ajouter
manuellement)}\label{captures-duxe9cran-ajouter-manuellement}

\begin{itemize}
\tightlist
\item
  Screenshot 1 : Interface principale
\item
  Screenshot 2 : Visualisation des classes d'obésité
\item
  Screenshot 3 : Grignotage vs obésité
\end{itemize}

\section{7. Références}\label{ruxe9fuxe9rences}

\begin{itemize}
\tightlist
\item
  \href{https://shiny.posit.co/}{Documentation Shiny}
\item
  \href{https://www.tidyverse.org/}{Tidyverse}
\item
  \href{https://www.kaggle.com/datasets/}{Dataset original - Kaggle}
\end{itemize}

\end{document}
